% !TEX root = ../thesis.tex
% створюємо вступ
\intro
\pagestyle{plain}

Проблеми, з якими людство може зіштовхнутися у майбутньому, цікавлять людину давно, а особливо --- з початком НТР та швидкого прогресу 
соціальних, економічних та наукових аспектів життя. Письменники (особливо письменники-фантасти) часто користуються 
свободою творення для побудови різних варіантів розвитку людства. Як приклад можна навести цикл оповідань Айзека 
Азімова "Я, робот" \cite{asimov1950robot}, у якому автор намагається осягнути питання, які виникнуть із появою розумних 
машин. 

Якщо звернутись до більш наукових джерел, можна теж побачити великий пласт різного ступеню підкріпленості роздумів про 
типи та розв'язки проблем майбутнього. Міждсциплінарну галузь, що має на меті дослідження майбутнього, називають 
\emph{футурологією}. Словник Merriam-Webster \cite{MerriamWebster2009} надає визначення футурології як науки, що 
розглядає майбутні можливості на основі теперішніх трендів (\emph{переклад авт.}).

Україна теж не залишається осторонь цього процесу. Сам факт фізичного
 розташування нашої країни у просторово-часовому континуумі робить
 настання майбутнього, а разом із тим пов'язаних проблем, неминучим. Залишається лише питання
 конкретного вигляду цих проблем та способів їхнього розв'язання.

 У даній роботі зроблено спробу огляду та систематизації українських національних проблем майбутнього.