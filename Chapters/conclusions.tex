%!TEX root = ../thesis.tex
% створюємо Висновки до всієї роботи
\conclusions

У рамках виконання роботи було виконано такі завдання:

\begin{itemize}
\item Оглянути наявні способи обчислення дискретного логарифму у групах заданого вигляду
\item Обрати оптимальний з точки зору часу роботи, простоти реалізації та здатності бути виконаним паралельно алгоритм
\item Продумати архітектуру та реалізацію алгоритму
\item Провести огляд сучасних платформ хмарних обчислень та обрати оптимальну.
\item Провести дослідження процесу виконання реалізації на обраній платформі, визначити оптимальні параметри для запуску.
\item Дослідити процес виконання з обраними параметрами. Оцінити розмір та вартість розв'язку проблеми дискретного логарифму, що можна вирішити даною реалізацією за період часу у один рік.
\item Зробити висновки щодо ефективності та доцільності даного підходу до проблеми дискретного логарифму.

\item Реалізувати кілька версій алгоритму, враховуючи обрану платформу.
\end{itemize}

Отримані результати свідчать про наступне:

\begin{itemize}
\item проблема дискретного логарифму може бути розв'язана у хмарній моделі ефективніше ніж у звичайній за рахунок паралельних обчислень
\item простий варіант алгоритму index-calculus вимагає значних матеріальних та часових ресурсів для проблем невеликого для сучасної криптографії розміру у 160-170 біт.
\end{itemize}

Напрямками подальшого дослідження можуть бути:

\begin{itemize}
\item проведення аналогічної роботи для інших варіантів того самого алгоритму
\item проведення аналогічної роботи для інших алгоритмів
\item проведення аналогічної роботи для інших мультиплікативних груп
\item дослідження інших підходів до паралелізації обчислення даної задачі у різних моделях
\end{itemize}